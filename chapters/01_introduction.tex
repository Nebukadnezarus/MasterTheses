% !TeX root = ../main.tex
% Add the above to each chapter to make compiling the PDF easier in some editors.

\chapter{Introduction}\label{chapter:introduction}

\section{Motivation}
Micro air vehicles (MAVs) with efficient autonomous navigation can strengthen applications such as search-and-rescue and last-mile delivery, where safety, robustness, and endurance are critical. Conventional quadrotors offer agility and precise control but are power-inefficient in sustained forward flight. Fixed-wing platforms are efficient but cannot hover and are less maneuverable in confined spaces. Hybrid vertical take-off and landing (VTOL) concepts improve mission versatility but increase mechanical and control complexity. This thesis investigates an intermediate design: a multirotor augmented with fixed aerodynamic surfaces to harvest passive lift during horizontal motion while keeping multirotor agility.

\section{Problem statement and scope}
In the presence of aerodynamic surfaces, hard-to-model aerodynamic forces become significant and challenge controller design. We aim to develop a platform and control strategy that:
\begin{itemize}
  \item preserves quadrotor agility while improving forward-flight efficiency via passive lift,
  \item tracks agile trajectories accurately without requiring aerodynamic parameter identification, and
  \item remains robust to modeling errors and disturbances.
\end{itemize}
The central question is whether an Incremental Nonlinear Dynamic Inversion (INDI)-based controller can achieve accurate trajectory tracking on an aerodynamic surface-enhanced quadrotor without an explicit aerodynamic model.

\section{Contributions}
This thesis presents:
\begin{itemize}
  \item a design of an aerodynamic surface-enhanced quadrotor in X-wing configuration,
  \item a dynamics model combining a standard quadrotor 6-DoF model with simplified quadratic lift/drag,
  \item a trajectory-tracking controller based on geometric control and INDI, including a coordinated-turn option, and
  \item an experimental evaluation: thrust-map identification, agility/controllability, aerodynamic disturbance characterization, and efficiency assessment.
\end{itemize}

\section{Thesis organization}
Chapter~\ref{chapter:related-work} reviews related work and motivates the chosen design. Chapter~\ref{chapter:dynamics-model} derives the model. Chapter~\ref{chapter:platform-design} details the platform. Chapter~\ref{chapter:control-architecture} presents the controller. Chapters~\ref{chapter:experimental-setup} and~\ref{chapter:experiments-evaluation} describe the setup and experiments. Chapter~\ref{chapter:conclusion} concludes.
