% !TeX root = ../main.tex

\chapter{Control Architecture}\label{chapter:control-architecture}

This chapter presents the flight control system used on the X-wing quadrotor.
The architecture combines (i) \emph{flatness-based reference generation} for dynamically
feasible trajectories, (ii) a \emph{geometric outer loop} on $SE(3)$ that turns
translational tracking errors into desired attitude and collective thrust, and
(iii) an \emph{Incremental Nonlinear Dynamic Inversion (INDI)} inner loop that
realizes the requested angular dynamics while rejecting unmodeled aerodynamic
effects from the wings.
The wings are \emph{not} explicitly modeled in the controller; instead, their influence
is treated as a disturbance and compensated incrementally by INDI
\cite{Smeur2016,Oosedo2017,vanKampen2018,Tzoumanikas2021}.

\section{Reference Generation}\label{sec:ref-gen}
Quadrotor dynamics are differentially flat with respect to
$\mathbf{p}_d=[x_d,y_d,z_d]^\top$ and $\psi_d$.
We generate $(\mathbf{p}_d,\dot{\mathbf{p}}_d,\ddot{\mathbf{p}}_d,\psi_d)$
from a minimum-snap polynomial or flatness-based planner; jerk/snap may be used internally
for smoothness but are not required by the controller interface.

\paragraph{Coordinated-turn option.}
In forward flight, sideslip can be bounded by aligning the body $x$-axis with the
horizontal velocity and setting a yaw-rate reference approximately as
\begin{equation}
\dot{\psi}_d \;\approx\; \frac{a_{y,d}}{V_d\cos\theta_d},
\end{equation}
where $a_{y,d}$ is the desired lateral acceleration, $V_d=\|\dot{\mathbf{p}}_d\|$, and
$\theta_d$ the pitch. We use this option when tracking fast, curving
trajectories; otherwise $\psi_d$ follows a commanded heading.

\section{Geometric Outer Loop}\label{sec:outer}
The outer loop maps position/velocity errors to a desired attitude $R_d\in SO(3)$
and a collective thrust $f_{\text{coll}}$.

\subsection{Translational feedback and force target}\label{sec:outer-force}
Let $\tilde{\mathbf{p}}=\mathbf{p}_d-\mathbf{p}$ and $\tilde{\mathbf{v}}=\dot{\mathbf{p}}_d-\mathbf{v}$.
With diagonal gains $\mathbf{k}_p,\mathbf{k}_v>0$ (elementwise),
\begin{equation}
\mathbf{a}_c \;=\; \mathbf{k}_p\odot\tilde{\mathbf{p}}
                  \;+\; \mathbf{k}_v\odot\tilde{\mathbf{v}}
                  \;+\; \ddot{\mathbf{p}}_d \;-\; \mathbf{g},
\label{eq:ac-nom}
\end{equation}
defines the nominal commanded acceleration in the world frame (ENU, $\mathbf{g}=[0,0,-g]^\top$).

\paragraph{Aerodynamic feedforward from onboard sensing.}
We estimate a body-frame specific force $\mathbf{a}_B$ from the IMU (bias-compensated)
or a model fallback and low-pass filter it, $\mathbf{a}_{B,f}=\mathrm{LPF}_a(\mathbf{a}_B)$.
From filtered motor speeds $\boldsymbol{\omega}_m$ we form
$\mathbf{f}_f = k_t\,\mathrm{LPF}_m(\boldsymbol{\omega}_m^{\circ 2})+\mathbf{b}$,
so that $f_{T,f}=\mathbf{1}^\top\mathbf{f}_f$.
A world-frame aero-compensation term follows as
\begin{equation}
\mathbf{a}_{\text{aero}} \;=\; R\!\left(\frac{f_{T,f}}{m}\,\mathbf{e}_3 - \mathbf{a}_{B,f}\right),
\qquad
\mathbf{a}_c \;\leftarrow\; \mathbf{a}_c + \mathbf{a}_{\text{aero}}
\;\;\text{if } z>0.5~\mathrm{m}.
\label{eq:aero-comp-outer}
\end{equation}
Intuitively, $R(f_{T,f}\mathbf{e}_3/m)$ is the thrust-induced specific force in the world;
subtracting the measured specific force yields the unmodeled aero contribution to be
fed forward.

\subsection{Attitude target and thrust projection}\label{sec:outer-att}
We set the desired body $z$-axis to the direction of the commanded acceleration,
$\mathbf{z}_B^d=\mathbf{a}_c/\|\mathbf{a}_c\|$.
For heading, we prefer a coordinated-turn alignment:
\[
\mathbf{x}_c =
\begin{cases}
\dot{\mathbf{p}}_d/\|\dot{\mathbf{p}}_d\|, & \|\dot{\mathbf{p}}_d\|>0.2~\mathrm{m/s},\\
R_z(\psi_d)\,\mathbf{e}_x, & \text{otherwise},
\end{cases}
\quad (\mathbf{x}_c)_z=0,\;\|\mathbf{x}_c\|=1.
\]
To avoid the degeneracy $\mathbf{z}_B^d\parallel\mathbf{x}_c$, we set
$\mathbf{x}_c\leftarrow -\mathbf{e}_z$ if $\angle(\mathbf{z}_B^d,\mathbf{x}_c)<5^\circ$.
Then
\begin{equation}
\mathbf{y}_B^d = \frac{\mathbf{z}_B^d\times\mathbf{x}_c}{\|\mathbf{z}_B^d\times\mathbf{x}_c\|},
\qquad
\text{flip } \mathbf{y}_B^d \text{ if } (\mathbf{y}_B^d)^\top(R\mathbf{e}_y)<0,
\qquad
\mathbf{x}_B^d = \frac{\mathbf{y}_B^d\times\mathbf{z}_B^d}{\|\mathbf{y}_B^d\times\mathbf{z}_B^d\|},
\end{equation}
and $R_d=[\,\mathbf{x}_B^d\;\mathbf{y}_B^d\;\mathbf{z}_B^d\,]$.
We compute collective thrust by projection to reduce transients:
\begin{equation}
f_{\text{coll}} \;=\; m\,\mathbf{a}_c^\top (R\mathbf{e}_3).
\label{eq:thrust-proj}
\end{equation}

\subsection{Tilt-prioritized attitude control and rate reference}\label{sec:tilt-prio}
Following the tilt-prioritized design of Föhn (2020), with attitude error
$q_e=q^{-1}\!\otimes q_d$ and gains $k_{\text{att},xy},k_{\text{att},z}$,
\begin{align}
T_{\text{att}} &= \mathrm{diag}(k_{\text{att},xy},k_{\text{att},xy},k_{\text{att},z}),
\\
\mathbf{t}(q_e) &= 
\begin{bmatrix}
q_{e,w}q_{e,x}-q_{e,y}q_{e,z}\\
q_{e,w}q_{e,y}+q_{e,x}q_{e,z}\\
q_{e,z}
\end{bmatrix},
\quad \text{if } q_{e,w}\le 0 \text{ then } t_3\leftarrow -t_3,
\\
\boldsymbol{\omega}_d &= \frac{2}{\sqrt{q_{e,w}^2+q_{e,z}^2}}\;T_{\text{att}}\;\mathbf{t}(q_e).
\label{eq:omega-d}
\end{align}
A simple rate-plus-error form produces an \emph{angular-acceleration proxy}
for INDI:
\begin{equation}
\boldsymbol{\alpha}_d \;=\; \boldsymbol{\omega}_d \;+\; 
\mathbf{k}_{\text{rate}}\odot(\boldsymbol{\omega}_d-\boldsymbol{\omega}).
\label{eq:alpha-d}
\end{equation}
We forward $(R_d,f_{\text{coll}},\boldsymbol{\alpha}_d)$ to the inner loop.

\section{INDI Inner Loop}\label{sec:indi}
The inner loop realizes the requested rotational dynamics robustly without an explicit
aerodynamic model. Over a small sampling interval $\Delta t$, the change in the measured
output $\mathbf{y}$ (specific force or angular acceleration proxy) satisfies
\begin{equation}
\Delta\mathbf{y} \;\approx\; B\,\Delta\mathbf{u},
\end{equation}
with $B$ the (local) control-effectiveness matrix and $\Delta\mathbf{d}$ (disturbance change)
second order in $\Delta t$; hence the incremental update largely cancels unmodeled
aerodynamics \cite{Smeur2016,Oosedo2017,vanKampen2018,Tzoumanikas2021}.

\subsection{Measured/filtered quantities}\label{sec:indi-sensed}
We use gyroscope rates $\boldsymbol{\omega}$ and their filtered derivative as an acceleration proxy,
\begin{equation}
\dot{\boldsymbol{\omega}}_f \;=\; \frac{\mathrm{d}}{\mathrm{d}t}\big(\mathrm{LPF}_\omega(\boldsymbol{\omega})\big),
\end{equation}
and reconstruct filtered rotor thrusts from motor speeds as in \eqref{eq:aero-comp-outer}. The
filtered rotor moments follow from the allocation matrix $G$:
\begin{equation}
\begin{bmatrix} f_{T,f}\\ \boldsymbol{\tau}_f \end{bmatrix} \;=\; G\,\mathbf{f}_f.
\label{eq:tau-f}
\end{equation}

\subsection{Nominal NDI versus incremental INDI moments}\label{sec:ndi-vs-indi}
Let $\boldsymbol{\mu}=[\mu_0,\mu_x,\mu_y,\mu_z]^\top=[f_T,\tau_x,\tau_y,\tau_z]^\top$.
We form two moment requests:
\begin{align}
\boldsymbol{\mu}_{\text{NDI}} &=
\begin{bmatrix}
m\,f_{\text{coll}}\\
(J\boldsymbol{\alpha}_d + \boldsymbol{\omega}\times J\boldsymbol{\omega})_x\\
(J\boldsymbol{\alpha}_d + \boldsymbol{\omega}\times J\boldsymbol{\omega})_y\\
(J\boldsymbol{\alpha}_d + \boldsymbol{\omega}\times J\boldsymbol{\omega})_z
\end{bmatrix},
\label{eq:mu-ndi}\\
\boldsymbol{\mu}_{\text{INDI}} &=
\begin{bmatrix}
m\,f_{\text{coll}}\\
\boldsymbol{\tau}_f + J\big(\boldsymbol{\alpha}_d - \dot{\boldsymbol{\omega}}_f\big)
\end{bmatrix}.
\label{eq:mu-indi}
\end{align}
Equation~\eqref{eq:mu-indi} is the incremental update: the moment change uses the mismatch
between desired and measured angular accelerations, injecting the \emph{measured} effectiveness
$J$ and $\boldsymbol{\tau}_f$.

\paragraph{Yaw-stability tweak.}
To avoid yaw oscillations observed in practice, we blend the two requests and set
\begin{equation}
\mu_z \;\leftarrow\; (\boldsymbol{\mu}_{\text{NDI}})_z,
\end{equation}
while keeping roll/pitch from \eqref{eq:mu-indi}.

\paragraph{Mode switch at low altitude.}
For takeoff and near-ground operation we disable INDI and use
$\boldsymbol{\mu}\leftarrow\boldsymbol{\mu}_{\text{NDI}}$ if $z<0.5$~m or a configuration flag is off.

\subsection{Update to rotor thrusts}\label{sec:indi-to-rotors}
With the inverse allocation matrix $G^{-1}$ (precomputed from geometry, including motor tilt),
the commanded rotor thrusts are
\begin{equation}
\mathbf{u} \;=\; G^{-1}\,\boldsymbol{\mu},
\qquad f_T=\mathbf{1}^\top\mathbf{u}.
\label{eq:indi-allocation}
\end{equation}
Each rotor thrust maps to speed via the quadratic thrust curve,
\begin{equation}
f_i \;=\; k_t\,\omega_i^2 \;+\; b_i,
\qquad
\omega_i \;=\; \sqrt{\max(0,(f_i-b_i)/k_t)}.
\label{eq:thrust-map}
\end{equation}
Commands are sent to the ESCs (DShot). Saturation and safety guards are applied at the actuation layer.

\section{Control Allocation and Constraints}\label{sec:allocation}
The allocation matrix $G\in\mathbb{R}^{4\times 4}$ maps individual rotor thrusts to total thrust
and body moments,
\begin{equation}
\boldsymbol{\mu}=G\,\mathbf{u},
\qquad
\mathbf{u}=[f_1,f_2,f_3,f_4]^\top,
\end{equation}
and encodes the arm geometry and $\SI{5}{\degree}$ motor tilt (see Chapter~\ref{chapter:experimental-setup}).
We use its inverse for real-time mapping \eqref{eq:indi-allocation}.
Commanded thrusts are clipped to $f_i\in[f_{\min},f_{\max}]$ and total thrust is bounded for hardware safety.
(When needed, a constrained least-squares allocator can replace $G^{-1}$ to enforce non-negativity and torque bounds.)

\section{Implementation Notes}\label{sec:impl-notes}
\paragraph{Filtering.}
First-order low-pass filters $\mathrm{LPF}_a,\mathrm{LPF}_\omega,\mathrm{LPF}_m$ use cutoffs
chosen below the control bandwidth to avoid amplifying sensor noise in $\dot{\boldsymbol{\omega}}_f$.
All filters are tuned consistently across the outer/inner loops.

\paragraph{Sensing and fusion.}
The controller runs with IMU rates at high frequency; position/velocity references come from Vicon or onboard state estimation.
Delays are kept small by using incremental updates; no explicit delay compensation was required.

\paragraph{Scheduling.}
The inner INDI loop runs faster than the outer geometric loop (rate separation).
Allocation and motor mapping execute at the inner-loop rate.

\paragraph{Logging and health.}
We log $\boldsymbol{\omega},\dot{\boldsymbol{\omega}}_f,\mathbf{f}_f,\boldsymbol{\tau}_f$ and the residual
$\boldsymbol{\tau}_f - J\dot{\boldsymbol{\omega}}_f - \boldsymbol{\omega}\times J\boldsymbol{\omega}$ to monitor effectiveness.
Guard checks reject invalid states or setpoints.

\section{Summary}\label{sec:ctrl-summary}
The proposed control system stacks a geometric $SE(3)$ outer loop with an INDI inner loop.
The outer loop produces $(R_d,f_{\text{coll}},\boldsymbol{\alpha}_d)$ by combining
proportional–derivative translational feedback with an aero feedforward term from onboard sensing,
and a coordinated-turn attitude construction.
The INDI layer then incrementally inverts the rotational dynamics using measured effectiveness,
blending a nominal NDI yaw to suppress oscillations.
This architecture achieves robust tracking across hover and forward-flight regimes while compensating
for the unmodeled aerodynamic influence of the X-wing surfaces.

% --- References used in this chapter ---
% Add to your .bib if not already present:
% \cite{Smeur2016,Oosedo2017,vanKampen2018,Tzoumanikas2021}
% (Optionally add Föhn 2020 for tilt-prioritized attitude control.)

