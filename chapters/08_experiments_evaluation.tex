% !TeX root = ../main.tex

\chapter{Experiments and Evaluation}\label{chapter:experiments-evaluation}

We design experiments to validate thrust mapping, agility/controllability, aerodynamic disturbance characterization, and efficiency gains.

\section{Thrust map identification}
- Throttle-to-thrust and RPM-to-thrust mapping; present identified curves and fitted models.

\begin{figure}[htbp]
  \centering
  \begin{tikzpicture}
  \begin{axis}[tumplot, xlabel={Throttle [\%]}, ylabel={Thrust [N]}]
      \addplot+[mark=*]
        table[x=throttle_pct, y=thrust_N, col sep=comma]{data/thrustmap_throttle_from_lookup_full_drone_2808.csv};
      \addlegendentry{Full drone (slice)}
    \end{axis}
  \end{tikzpicture}
  \caption{Throttle–thrust curve derived from the thrust lookup at a fixed voltage slice.}
\end{figure}

\section{Agility and tracking}
- 3g circles in flight arena; metrics: RMS position/attitude tracking error, control effort.

\section{Aerodynamic forces/moments quantification}
- Identify equivalent $k_D,k_L$ or $C_L(\alpha), C_D(\alpha)$ from maneuvering flight data; discuss sensitivity.

\section{Efficiency assessment}
- Compare theoretical quad thrust input vs. measured thrust input in AIDA hall flights; quantify energy savings due to passive lift.

\section{Baseline comparison}
- Compare against baseline quadrotor controller without INDI or without wings.
