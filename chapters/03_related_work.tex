% !TeX root = ../main.tex

\chapter{Related Work}\label{chapter:related-work}

We review UAV archetypes and control approaches with emphasis on agility, efficiency, and controllability, motivating our X-wing multirotor with passive aerodynamic lift and an INDI-based controller.

\section{UAV archetypes and trade-offs}
- Quadrotors: highly agile and controllable, hover capable, but power-inefficient in high-speed flight.
- Fixed-wing: efficient at speed and range, but poor agility in confined spaces; requires runway/launch.
- Tailsitters/tilt-rotors (VTOL): combine hover and efficient cruise; transitions and cross-couplings complicate control.
- Aerodynamic-surface-enhanced multirotors: retain quad agility while harvesting passive lift in forward flight.

\section{Control approaches}
- Geometric control on SE(3) and differential flatness-based planners for agile trajectory tracking.
- Disturbance observers, incremental (nonlinear) dynamic inversion, and adaptive methods for model uncertainties.
- INDI in particular achieves robustness by leveraging incremental relations between actuator changes and measured accelerations/ang. rates, reducing dependence on aerodynamic models.

\section{Positioning our work}
We target accurate tracking during agile maneuvers without modeling aerodynamic coefficients, enabled by INDI, while leveraging passive lift for efficiency during forward flight. This guides design choices in Chapters~\ref{chapter:platform-design} and~\ref{chapter:control-architecture}.
