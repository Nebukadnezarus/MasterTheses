% !TeX root = ../main.tex

\chapter{Dynamics Model}\label{chapter:dynamics-model}

This chapter introduces the six-degree-of-freedom (6-DoF) rigid-body model used to describe the motion of the X-wing quadrotor platform. The formulation captures the essential translational and rotational dynamics with lumped aerodynamic effects from the integrated wing surfaces. The model is intended to provide a physically consistent basis for later control design and parameter identification.

\section{Coordinate Frames and States}
The equations of motion are expressed in the world (inertial) and body-fixed frames.

\begin{itemize}
    \item The \textbf{world frame} follows the ENU convention, with gravity $\mathbf{g} = [0,\,0,\,-g]^\top$.
    \item The \textbf{body frame} is located at the vehicle's center of gravity (CoG), with the $x$-axis pointing forward, $y$-axis to the left, and $z$-axis upward, normal to the rotor plane.
\end{itemize}

The system states are
\begin{equation}
(\mathbf{p},\,\mathbf{v},\,R,\,\boldsymbol{\omega}) 
    \in \mathbb{R}^3 \times \mathbb{R}^3 \times SO(3) \times \mathbb{R}^3,
\end{equation}
where $\mathbf{p}$ and $\mathbf{v}$ denote position and linear velocity in the world frame, $R \in SO(3)$ is the rotation matrix from body to world frame, and $\boldsymbol{\omega}$ is the body angular velocity.  
The control inputs are the individual rotor thrusts
\begin{equation}
\mathbf{u} = [f_1,\,f_2,\,f_3,\,f_4]^\top,
\end{equation}
each acting approximately along the body $+z$ direction.  
Explicit control allocation and motor dynamics are treated in Chapter~\ref{chapter:control-architecture}.

\section{Forces}
The total force acting on the vehicle in the world frame is the sum of gravity, rotor thrust, and aerodynamic forces:
\begin{equation}
\mathbf{F} = m\mathbf{g} + \mathbf{F}_T + R\,\mathbf{F}_\text{aero}.
\end{equation}

\subsection{Gravity}
The gravitational force is given by $m\mathbf{g}$ with $m$ the total mass of the system.

\subsection{Rotor Thrust}
Each rotor $i$ produces a thrust $f_i$ approximately aligned with the body $+z$ axis.  
The total thrust in the world frame is
\begin{equation}
\mathbf{F}_T = \bigg(\sum_{i=1}^{4} f_i\bigg) R\,\mathbf{e}_3.
\end{equation}

\subsection{Aerodynamic Forces}
The fixed aerodynamic surfaces generate lift and drag forces that depend on the body-frame airspeed
\begin{equation}
\mathbf{v}_a = R^\top(\mathbf{v} - \mathbf{v}_w),
\end{equation}
where $\mathbf{v}_w$ is the wind velocity in the world frame.  
For design and performance estimation, the aerodynamic forces are computed using standard expressions:
\begin{align}
L &= \tfrac{1}{2}\rho S C_L(\alpha) \|\mathbf{v}_a\|^2, & D &= \tfrac{1}{2}\rho S C_D(\alpha) \|\mathbf{v}_a\|^2,
\end{align}
where $\rho$ is air density, $S$ is the total wing area, and $C_L(\alpha)$ and $C_D(\alpha)$ are the lift and drag coefficients as functions of angle of attack $\alpha$.
We use the approximations $C_L = a_\alpha \alpha$ and $C_D = C_{D0} + k C_L^2$ to estimate the wing contributions during the platform design phase.

However, these aerodynamic forces are \textbf{not explicitly modeled in the controller}.
Instead, they are treated as external disturbances that are automatically rejected by the incremental nonlinear dynamic inversion (INDI) controller presented in Chapter~\ref{chapter:control-architecture}.
This approach avoids the need for precise aerodynamic parameter identification or online angle-of-attack estimation, relying on the controller's inherent robustness to unmodeled dynamics.

\section{Moments}
The moments acting on the vehicle are composed of rotor-generated torques and lumped aerodynamic moments:
\begin{equation}
\boldsymbol{\tau} = \boldsymbol{\tau}_T + \boldsymbol{\tau}_\text{aero}.
\end{equation}
The rotor moments $\boldsymbol{\tau}_T$ arise from thrust lever arms and counter-torques of the propellers, while $\boldsymbol{\tau}_\text{aero}$ accounts for small contributions from the aerodynamic center offset and asymmetric lift distribution.
The detailed mapping from individual rotor thrusts to force and moments (control allocation) is treated in Chapter~\ref{chapter:control-architecture}.

\section{Equations of Motion}
The rigid-body dynamics of the vehicle are expressed as
\begin{align}
\dot{\mathbf{p}} &= \mathbf{v}, \\
\dot{\mathbf{v}} &= \tfrac{1}{m}\big( m\mathbf{g} + \mathbf{F}_T + R\,\mathbf{F}_\text{aero} \big), \\
\dot{R} &= R[\boldsymbol{\omega}]_\times, \\
J\dot{\boldsymbol{\omega}} &= -\boldsymbol{\omega}\times J\boldsymbol{\omega} + \boldsymbol{\tau},
\end{align}
where $J$ is the inertia matrix expressed in the body frame, and $[\boldsymbol{\omega}]_\times$ denotes the skew-symmetric matrix representation of the cross product.

\section{Assumptions and Simplifications}
The following assumptions are made to keep the model tractable:
\begin{itemize}
    \item Quasi-steady aerodynamics; unsteady and dynamic-stall effects are neglected.
    \item No prop-wash coupling or aerodynamic interference between rotors and wings.
    \item Symmetric mass and inertia distribution with negligible structural flexibility.
    \item Fixed-pitch rotors; gyroscopic and blade flapping effects are ignored.
    \item Aerodynamic forces and moments are treated as lumped, quasi-linear disturbances for identification.
\end{itemize}

\section{Remarks}
The presented model provides a concise yet comprehensive description of the platform's motion.  
It captures the key couplings between thrust, gravity, and aerodynamic effects while remaining simple enough for real-time simulation and control.  
In later chapters, this model forms the foundation for the geometric and incremental nonlinear dynamic inversion (INDI) control strategies, which compensate for the unmodeled aerodynamic disturbances in flight.

